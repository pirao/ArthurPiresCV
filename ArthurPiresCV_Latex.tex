%------------------------
% Resume Template
% Author : Anubhav Singh
% Github : https://github.com/xprilion
% License : MIT
%------------------------

\documentclass[a4paper,20pt]{article}

\usepackage{latexsym}
\usepackage[empty]{fullpage}
\usepackage{titlesec}
\usepackage{marvosym}
\usepackage[usenames,dvipsnames]{color}
\usepackage{verbatim}
\usepackage{enumitem}
\usepackage[pdftex]{hyperref}
\usepackage{fancyhdr}

\pagestyle{fancy}
\fancyhf{} % clear all header and footer fields
\fancyfoot{}
\renewcommand{\headrulewidth}{0pt}
\renewcommand{\footrulewidth}{0pt}

% Adjust margins
\addtolength{\oddsidemargin}{-0.530in}
\addtolength{\evensidemargin}{-0.375in}
\addtolength{\textwidth}{1in}
\addtolength{\topmargin}{-.45in}
\addtolength{\textheight}{1in}

\urlstyle{rm}

\raggedbottom
\raggedright
\setlength{\tabcolsep}{0in}

% Sections formatting
\titleformat{\section}{
  \vspace{-10pt}\scshape\raggedright\large
}{}{0em}{}[\color{black}\titlerule \vspace{-6pt}]

%-------------------------
% Custom commands
\newcommand{\resumeItem}[2]{
  \item\small{
    \textbf{#1}{#2 \vspace{-2pt}}
  }
}

% \newcommand{\resumeItemWithoutTitle}[1]{
%   \item\small{
%     {\vspace{-2pt}}
%   }
% }

\newcommand{\resumeSubheading}[4]{
  \vspace{-1pt}\item
    \begin{tabular*}{0.97\textwidth}{l@{\extracolsep{\fill}}r}
      \textbf{#1} & #2 \\
      \textit{#3} & \textit{#4} \\
    \end{tabular*}\vspace{-5pt}
}


\newcommand{\resumeSubItem}[2]{\resumeItem{#1}{#2}\vspace{-3pt}}

\renewcommand{\labelitemii}{$\circ$}

\newcommand{\resumeSubHeadingListStart}{\begin{itemize}[leftmargin=*]}
\newcommand{\resumeSubHeadingListEnd}{\end{itemize}}
\newcommand{\resumeItemListStart}{\begin{itemize}}
\newcommand{\resumeItemListEnd}{\end{itemize}\vspace{-5pt}}

%-----------------------------
%%%%%%  CV STARTS HERE  %%%%%%


\begin{document}

%----------HEADING-----------------
\begin{tabular*}{\textwidth}{l@{\extracolsep{\fill}}r}
  \textbf{{\LARGE Arthur Cancellieri Pires}} & Email: \href{mailto:arthur.cancellieri.pires@gmail.com}{arthur.cancellieri.pires@gmail.com}\\
  \href{https://www.linkedin.com/in/arthur-cancellieri-pires-730963160}{LinkedIn: Arthur Cancellieri Pires} & Mobile:~~~+55-27-99986-2672\\
  \href{https://github.com/pirao}{GitHub: ~@Pirao} \\
\end{tabular*}


\section{Professional Summary}
\resumeSubHeadingListStart
  \resumeSubItem{}{AI research lead focused on predictive maintenance from sensor time-series, with 5+ years building ML pipelines and deep learning models for anomaly detection in condition monitoring applications.}
  \resumeSubItem{}{Experienced in Python + PyTorch and data workflows at scale (30k+ windows/day), translating model outputs into diagnostic tools used by maintenance teams.}
\resumeSubHeadingListEnd
\vspace{-5pt}

%----------Work Experience-----------------

\section{Work Experience}
  \resumeSubHeadingListStart
    \resumeSubheading{Railway Laboratory (Lafer)}{On-site}
    {AI Research Lead (PhD Candidate) -- VALE partnership}{January 2022 - Present}
    \resumeItemListStart

        \resumeItem{Sensor Defect Detection: }
          {Developed deep learning models and batch checks to detect sensor faults and data-quality anomalies, achieving 91\% precision on internally labeled events and reducing data-issue detection latency from 3 months (manual review) to 1 day; tuned and compared model variants via Optuna studies.}

        \resumeItem{Track Defect Detection: }
          {Built transformer-based reconstruction models and a repeatability metric that converts residuals into calibrated anomaly probabilities to score repeatable, location-level defect candidates with sparse ground truth.}

        \resumeItem{Model Diagnostics (Weak Labels): }
          {Built a diagnostic workflow using weak labels (data-quality score) and latent-space analysis to interpret anomaly clusters, linking them to specific sensors/channels and response signatures to support maintenance investigation.}

        \resumeItem{Technical Leadership: }
          {Served as the technical lead for VALE's predictive maintenance program using IRVs, translating stakeholder needs into ML research deliverables, and coordinating technical execution across a 10-person team (undergraduate, Master's, and PhD).}

      \resumeItemListEnd
\vspace{2pt}

    \resumeSubheading{Railway Laboratory (Lafer)}{On-site}
    {AI Researcher (M.Sc. program) -- VALE partnership}{June 2020 - January 2022}
        \resumeItemListStart

        \resumeItem{ML Pipeline: }
          {Designed and operated a daily pipeline to ingest raw instrumented railway vehicle (IRV) data, engineer features, and generate model-ready windows for ML (30k+ windows/day); maintained the codebase in Git and reviewed outputs with maintenance engineers.}

        \resumeItem{Supervised Learning (Regression): }
          {Developed deep learning models to estimate vertical track geometry from IRV data, achieving 98\% $R^{2}$ on a held-out dataset.}

        % \resumeItem{Stakeholder Collaboration: }
        %   {Presented results and diagnostic analyses in recurring reviews with maintenance engineers and research stakeholders; incorporated feedback into data checks and model evaluation.}

        \resumeItemListEnd

\vspace{2pt}

    \resumeSubheading{Laboratory of Railway Dynamics and Tribology (LabTDF)}{On-site}
    {Undergraduate Researcher (B.Sc.) -- VALE partnership}{Jan 2019 - June 2020}
        \resumeItemListStart

        \resumeItem{Optimization (Genetic Algorithms): }
          {Developed a wheel--rail profile optimization methodology using measurement data and genetic algorithms; field-tested a wear-optimized profile, reducing wear by 20\% while maintaining fatigue performance and improving the L/V ratio by $\sim$35\%.}

        \resumeItem{Data Analysis / Decision Support: }
          {Analyzed wheel reprofiling limits and recommended changing the EFVM threshold from 3 mm to 2 mm based on the wheel-profile optimization study, increasing projected wheel life by at least 29\% (current profile) and 50\% (proposed); recommendation adopted.}

        \resumeItemListEnd
    \resumeSubHeadingListEnd


\vspace{-7pt}

\section{Projects}
\resumeSubHeadingListStart
  \resumeSubheading
    {Model Deployment Demo: Time-Series Anomaly Inference Service}{\href{https://github.com/pirao/<REPO>}{GitHub}}
    {FastAPI + Docker (personal project)}{2025}
    \resumeItemListStart
      \resumeItem{Serving API: }
        {Implemented a FastAPI service for time-series anomaly inference with input validation, model versioning, and JSON responses.}
      \resumeItem{Deployment: }
        {Containerized the service with Docker and documented local + container runs (docker build/run or docker compose) with example requests.}
      \resumeItem{Results + Reproducibility: }
        {Included an anonymized sample dataset and a reproducible evaluation script/notebook; reported baseline metrics and example outputs in the README.}
    \resumeItemListEnd
\resumeSubHeadingListEnd

\vspace{-5pt}

\section{Skills Summary}
  \resumeSubHeadingListStart
    \resumeSubItem{Programming}{~Python, SQL, MATLAB}
    \resumeSubItem{Frameworks}{~~~~PyTorch, PyTorch Lightning, scikit-learn, Optuna, pandas, NumPy, SciPy}
    \resumeSubItem{Tools}{~~~~~~~~~~~~~~Git, PostgreSQL, MySQL, SQLite}
    \resumeSubItem{ML Domains}{~~Time-series modeling, anomaly detection, remaining useful life estimation, feature engineering}
    \resumeSubItem{Languages}{~~~~~~Portuguese (Native), English (Fluent)}
  \resumeSubHeadingListEnd


\vspace{-5pt}

%-----------EDUCATION-----------------
\section{~~Education}
  \resumeSubHeadingListStart
    \resumeSubheading
      {State University of Campinas (UNICAMP)}{Campinas, Brazil}
      {PhD in Mechanical Engineering}{Expected February 2026}
      % {\scriptsize \textit{ \footnotesize{\newline{}\textbf{Research focus} Deep-learning-based track defect detection for railway condition monitoring using instrumented railway vehicle data}}}
    \resumeSubheading
      {State University of Campinas (UNICAMP)}{Campinas, Brazil}
      {Master's in Mechanical Engineering}{Jan 2022}
      % {\scriptsize \textit{ \footnotesize{\newline{}\textbf{Research focus} Deep-learning-based estimation of vertical track geometry from instrumented railway vehicle data}}}
    \resumeSubheading
      {Federal University of Espírito Santo (UFES)}{Vitória, Brazil}
      {Bachelor's Degree in Mechanical Engineering}{May 2020}
      % {\scriptsize \textit{ \footnotesize{\newline{}\textbf{Research focus} Machine-learning-based prediction of wheel--rail L/V ratio to assess margin to Nadal’s derailment criterion}}}
    \resumeSubHeadingListEnd

\vspace{-5pt}

\end{document}
